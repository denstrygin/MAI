\documentclass[12pt,a5paper]{book}
\usepackage[utf8]{inputenc}
\usepackage[russian]{babel}
%дополнительные символы для математики
\usepackage{amssymb}
%отступы для страницы
\usepackage{geometry}
\geometry{left=1.5cm}
\geometry{right=1.5cm}
\geometry{top=2.5cm}
\geometry{bottom=1cm}
%установка отступов для текста
\def \tab{3ex}
\setlength{\parindent}{\tab}
\setlength{\parskip}{0em}
% настройки колонтитула
\usepackage{fancyhdr}
\setcounter{page}{278}
\fancyhead{}
\fancyhead[LE,RO]{\thepage}
\fancyhead[CO]{\textit{Круговые функции}}
\fancyhead[CE]{\textit{Глава 19}}
\fancyfoot{}
\renewcommand{\footrulewidth}{0.0 mm}% толщина отделяющей полоски сверху
\renewcommand{\headrulewidth}{0.2 mm} %толщина отделяющей полоски снизу (её нет)
\fancyfootoffset[R]{0cm} %линия во всю страницу

\pagestyle{fancy}
\begin{document}
    \textbf{Теорема 318. } \textit{Для $|x| \leqslant 1$ существует точно одно
    $\nu$ такое, что}
    \begin{center}
        $cos\,y=x, \;\;\;\;\;\;\;\;\; 0\,{\leqslant}\,y\,{\leqslant}\,\pi,$
    \end{center}
    \textit{а именно,}
    \begin{center}
        $\displaystyle\nu=\frac{\pi}{2}-arc\;sin{x}.$
    \end{center}
    
    Д\,о\,к\,а\,з\,а\,т\,е\,л\,ь\,с\,т\,в\,о. Требования
    \begin{center}
        $cos\,y=x, \;\;\;\;\;\;\;\;\; 0\,{\leqslant}\,y\,{\leqslant}\,\pi$
    \end{center}
    равносильны требованиям
    \begin{center}
        $\displaystyle sin(\frac{\pi}{2}-y)=x\;\;\;\;\;\;\;\;\; -\frac{\pi}{2}\,{\leqslant}\,\frac{\pi}{2}-y\,{\leqslant}\,\frac{\pi}{2},$
    \end{center}
    и, следовательно,
    \begin{center}
        $\displaystyle\frac{\pi}{2}-y=arc\,sinx.$
    \end{center}

    \textbf{Определение 74.} arc\,cos \textit{x для $|x|\,{\leqslant}\,1$ есть y из теоремы }318.
    
    arc\,cos читается: арккосинус (или аркус косинус).
    
    \textbf{Теорема 319. }$\displaystyle\frac{d\,arc\,cosx}{dx}=-\frac{1}{1-x^2}$ \textit{для} $|x|<1$.
   
    Д\,о\,к\,а\,з\,а\,т\,е\,л\,ь\,с\,т\,в\,о. Из теорем 318 и 317 следует
    \begin{center}
        $(arc\,cosx)'=(\displaystyle\frac{\pi}{2}-arc\,sinx)'=-(arc\,sinx)'=-\frac{1}{1-x^2}$.
    \end{center}
    
    \textbf{Теорема 320. }\textit{Для каждого x существует точно одно y такое, что}
    \begin{center}
        $\displaystyle tg\,y=x, \;\;\;\;\;\;\;\; -\frac{\pi}{2}<y<\frac{\pi}{2}$
    \end{center}
    \textit{а именно,}
    \begin{center}
        $\displaystyle\nu=arc\;sin{\frac{x}{1+x^2}}.$
    \end{center}
    
    Д\,о\,к\,а\,з\,а\,т\,е\,л\,ь\,с\,т\,в\,о. 1) Для $-\displaystyle\frac{\pi}{2}<y<\frac{\pi}{2}$ имеем
    \begin{center}
        $(tg\,y)'=\displaystyle\frac{1}{cos^2y}>0.$
    \end{center}

    \newpage
    \pagestyle{fancy}
    \hspace{-\tab}Следовательно, может существовать, самое большое, одно требуемое y.
    
    2)Так как
    \begin{center}
        $\displaystyle|\frac{x}{\sqrt{1+x^2}}|=\sqrt{\frac{x^2}{1+x^2}}<1$,
    \end{center}
    
    \hspace{-\tab}то для
    \begin{center}
        $\displaystyle\nu=arc\;sin{\frac{x}{\sqrt{1+x^2}}}$
    \end{center}
    
    \hspace{-\tab}имеем
    \begin{center}
        $|y|<\displaystyle\frac{\pi}{2},$\\
        $cos\,y>0,$\\
        $sin\,y=\displaystyle\frac{x}{\sqrt{1+x^2}},$\\
        $cos^2y=1-sin^2y=1-\displaystyle\frac{x^2}{1+x^2}=\frac{1}{1+x^2},$\\
        $cos\,y=\displaystyle\frac{1}{\sqrt{1+x^2}},$\\
        $tg\,y=\displaystyle\frac{sin\,y}{cos\,y}=x.$
    \end{center}
    
    \textbf{Определение 75. }$arc\;tg\;x$ \textit{есть из теоремы} 320.\\
    arc\,tg читается: аркангес (или аркус тангенс).
    
    \textbf{Теорема 321.  } $\displaystyle\frac{d\,arc\,tgx}{dx}=\frac{1}{1+x^2}$
    
    Д\,о\,к\,а\,з\,а\,т\,е\,л\,ь\,с\,т\,в\,о. Из теорем 320 и 317 следует
    \begin{center}
        $(arc\,tgx)'=(arc\,sin\displaystyle\frac{x}{\sqrt{1+x^2}})=\frac{1}{\sqrt{1-\frac{x^2}{1+x^2}}}(\frac{x}{\sqrt{1+x^2}})'=$\\
        $\displaystyle=\sqrt{1+x^2}\frac{\sqrt{1+x^2}-\frac{x^2}{\sqrt{1+x^2}}}{1+x^2}=\frac{1}{1+x^2}$
    \end{center}
    
    \textbf{Теорема 322. }\textit{Для каждого x существует точно одно y такое, что}
    \begin{center}
        $ctg\,y=x, \;\;\;\;\;\;\;\;\; 0<y<\pi,$
    \end{center}
\end{document}
\documentclass{article}
\usepackage[utf8]{inputenc}
\usepackage[russian]{babel}

\usepackage{sagetex}
\setlength{\sagetexindent}{10ex}

\usepackage[left=15mm,top=30mm,bottom=30mm,right=15mm]{geometry}

\begin{document}
	\begin{center}
		\bfseries
		
		{\Large Московский авиационный институт\\ (национальный исследовательский университет)
			
		}
		
		\vspace{48pt}
		
		{\large Факультет информационных технологий и прикладной математики
		}
		
		\vspace{36pt}
		
		{\large Кафедра вычислительной математики и~программирования
			
		}
		
		
		\vspace{48pt}
		
		Лабораторная работа \textnumero 2 по курсу \enquote{Системы аналитических вычислений}
		
	\end{center}
	
	\vspace{350pt}
	
	\begin{flushright}
		\begin{tabular}{rl}
			Студент: &  Стрыгин Д.\,Д.\\
			Преподаватель: &  Гавриш О.\,Н.\\
			Группа: & М8О-206Б-19 \\
			Номер в списке: & 25 \\
		\end{tabular}
	\end{flushright}
	
	\vfill
\pagebreak

Уравнение фигуры:
$$f = -9*x^2 + 7*y^2 + 8*y*z - 3*z^2 - 4*x + 9*y - 10$$

\section{Построение исходной поверхности}
\begin{sagesilent}
var("x y z")
\end{sagesilent}
\begin{sageblock}
f(x, y, z) = -9*x^2 + 7*y^2 + 8*y*z - 3*z^2 - 4*x + 9*y - 10
\end{sageblock}

Выведем считанную функцию на экран:

\begin{center}
$\sage{f(x=x, y=y, z=z)}$
\end{center}

Построим исходную поверхность.

\begin{center}
\sageplot{implicit_plot3d(f, (x, -10, 10), (y, -10, 15), (z, -10, 10), figsize = 4.3)}
\end{center}

\section{Приведение поверхности к каноническому виду}
Составим матрицу из квадратичной формы нашей функции от трёх переменных.

\begin{sageblock}
A = Matrix([[-9, 0, 0]
            , [0, 7, 4]
            , [0, 4, -3]])
\end{sageblock}

Найдем собственные значения матрицы А.

\begin{sagesilent}
var("l")
\end{sagesilent}

\begin{sageblock}
E = matrix([[1, 0, 0]
            , [0, 1, 0]
            , [0, 0, 1]])

eigen_values = []
for eigen_value in solve((A - l * E).det() == 0, l):
    eigen_values.append(eigen_value.rhs())

\end{sageblock}

Собственные значения:
$$\sage{eigen_values[0]}$$
$$\sage{eigen_values[1]}$$
$$\sage{eigen_values[2]}$$

СЗ в численном виде: 
$$\sage{eigen_values[0].n()}$$
$$\sage{eigen_values[1].n()}$$
$$\sage{eigen_values[2].n()}$$

Теперь найдем собственные значения через встроенную функцию.

\begin{sageblock}
ev = A.eigenvalues()
a = ev[0].n()
b = ev[1].n()
c = ev[2].n()
\end{sageblock}

Получили: 
$$a = \sage{a}$$
$$b = \sage{b}$$
$$c = \sage{c}$$ 

Собственные значения, полученные при помощи встроенной в Sage функции, совпадают с найденными СВ при помощи единичной матрицы.

\newpage
Теперь найдём свободный член в каноническом виде

\begin{sageblock}
c1 = diff(f(x, y, z), x)
c2 = diff(f(x, y, z), y)
c3 = diff(f(x, y, z), z)
a_0 = f(-2/9, -27/74, -18/37)
\end{sageblock}

$\sage{solve([c1==0, c2==0,c3==0], x, y, z)}$

$$a0 = f(\frac{-2}{9}, \frac{-27}{74}, \frac{-18}{37}) = \sage{f(=2/9, -27/74, -18/37)}$$


Канонический вид $ax^2 + by^2 + cz^2 + a_0$

\begin{sageblock}
f_k = a * x**2 + b * y**2 + c * z**2 + a_0
\end{sageblock}

Получим уравнение
\begin{center}
$\sage{f_k(x=x, y=y, z=z)}$
\end{center}

Построим полученную поверхность
\begin{center}
  \sageplot{implicit_plot3d(f_k, (x, -10, 10), (y, -10, 15), (z, -10, 10), figsize = 4)}
\end{center}

\end{document}
\section{Раздел 2. Анализ миссии, целей и задач.}

Предприятие специализируется на разработке и производстве ракетных двигателей, которые вывели в космос первый искусственный спутник Земли, первого человека, первый самоходный аппарат «Луноход-1», ракетно-космическую систему «Энергия-Буран», новую российскую ракету-носитель «Ангара». Двигатели, разработанные предприятием, устанавливаются практически на все российские ракеты-носители: «Союз», «Протон», «Ангара» и на американские «Атлас-5» и «Антарес». Является разработчиком самого мощного в мире жидкостного ракетного двигателя РД-170/РД-171.

Каждая третья ракета-носитель в мире стартует на двигателях разработки и производства НПО «Энергомаш». За всю историю освоения человеком космического пространства, двигатели НПО «Энергомаш» осуществили более 2400 абсолютно успешных пусков. НПО «Энергомаш» выпустило более 12 000 двигателей, разработав более 60 различных модификаций двигательных установок первой и второй ступени.

Основными целями компании являются:
\begin{enumerate}
    \item Долгосрочные:
    \begin{itemize}
        \item aefae
        \item aefae
    \end{itemize}
    \item Краткосрочные:
    \begin{itemize}
        \item aefae
    \end{itemize}
\end{enumerate}

Основными задачами компании являются:
\begin{enumerate}
    \item Производственные:
    \begin{itemize}
        \item aefae
        \item aefae
    \end{itemize}
    \item Экспортные:
    \begin{itemize}
        \item aefae
    \end{itemize}
    \item Послепродажного обслуживания:
    \begin{itemize}
        \item aefae
    \end{itemize}
\end{enumerate}

\pagebreak
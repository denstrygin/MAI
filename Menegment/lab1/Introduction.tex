\section{Введение}

Российские отраслевые научные центры и предприятия занимают лидирующие мировые позиции в области космических разработок и их практической реализации. В период с 1957 года до нашего времени в СССР и РФ осуществлено свыше 3000 запусков космических аппаратов как гражданского, так и военного назначения. Подавляющее большинство завершилось успешным выводом на околоземную орбиту спутников, пилотируемых и беспилотных космических кораблей, прочих аппаратов.

Ракетная промышленность – наиболее сложная и наукоемкая из отраслей машиностроения. К ней относятся производственные предприятия, научные институты, исследовательские и испытательные центры. На сегодняшний день зарегистрировано свыше 100 фирм, на которых занято около 250 тысяч человек. Несмотря на наличие существенных проблем, ракетно-космическая промышленность в последние годы активно развивается, благодаря чему осуществляется поэтапное введение в эксплуатацию новой техники и реализуются программы освоения космоса.

В России ведется активная работа над созданием новой ракеты-носителя среднего класса «Союз-5», которая может вывести на орбиту перспективный пилотируемый корабль «Федерация» в 2022 году. Тяжелые ракеты «Ангара-А5» в ближайшие годы будут задействованы в запусках 600 спутников системы «Сфера». Научно-производственное объединение (НПО) Энергомаш ведет работы над новыми двигателями для обоих типов ракет - РД-171МВ и РД-191М.

----------------------------

Развитие современной экономики России находиться под влиянием деятельности предприятий оборонно-промышленного комплекса, к числу которых относят предприятия отрасли приборной промышленности — отрасль машиностроения, в которой осуществляются разработка и производство средств измерений обработки и представления информации, автоматических и автоматизированных систем управления.

Приборостроение является одной из наиболее прибыльных и в то же время наиболее капиталоемких отраслей машиностроения. Немногие страны мира, из числа наиболее развитых государств, обладают полным циклом (макротехнологии) создания аналогичной техники — подобную промышленность имеют 5-6 государств, обладающие высокими технологиями.

Субсидии предоставляются в рамках подпрограммы "Авиационные агрегаты и приборы" государственной программы «Развитие авиационной промышленности на 2013–2025 годы». Общая сумма субсидирования составила: 28,5 млрд. рублей, из них за период 2013-2021 освоено 18,2 млрд. рублей (стр.7).

Предусмотрено субсидирование расходов на разработку технических проектов, приобретение и изготовление технологического и испытательного оборудования, изготовление и испытание опытных образцов, оплату услуг по сертификации и регистрации результатов интеллектуальной деятельности, частичную оплату работ организаций-соисполнителей и других расходов.

К числу особенностей такого предприятия относят такие особенности:

\pagebreak